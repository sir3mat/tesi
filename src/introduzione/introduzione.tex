\chapter{Introduzione}
\section{Obiettivo}
Obiettivo della tesi è l'analisi di una piattaforma distribuita cloud based.

Lo scopo del progetto è ottimizzare il processo di sviluppo software
per l'erogazione dei servizi aziendali.
Questo ha portato alla progettazione e realizzazione di una piattaforma in grado di erogare
i servizi tipicamente richiesti in ambito aziendale e con la possibilità di integrare
personalizzazioni in base alle esigenze del business della clientela.
Si potrebbe quindi considerare la piattaforma come un boilerplate o un template
riutilizzabile che permette l'erogazione di nuovo software ai clienti in maniera più efficente.

Per ottimizzare il processo di sviluppo software sono state utilizzate tecnologie innovative,
è stato adottato un metodo di lavoro in grado di erogare software di qualità in breve tempo ed
è stata progettata un'architettura scalabile e distribuita per poter supportare al meglio le operazioni
nei vari contesti di utilizzo.

La realizzazione della piattaforma è avvenuta durante il periodo di tirocinio presso Fama Labs,
un'azienda erogatrice di servizi cloud innovativi.


\section{Panoramica}
La restante parte del documento vuole analizzare la piattaforma realizzata, le tecnologie e il metodo di lavoro.

Il capitolo successivo fornisce una panoramica del sistema software
e verrà presentato il metodo di lavoro applicato. Nel terzo capitolo verranno presentati
gli strumenti tecnologici utilizzati per lo sviluppo della piattaforma. Il quarto capitolo
analizza nel dettaglio i componenti principali della piattaforma. Il capitolo successivo
vuole invece soffermarsi sulle fasi di integrazione e deployment del software.
