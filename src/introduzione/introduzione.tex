\chapter{Introduzione}
\section{Obiettivo}
Oggi moltissime aziende e professionisti hanno la necessità di costruire
una piattaforma software in grado di supportare le esigenze del business dell'organizzazione e
ottimizzare i propri processi di lavoro. Questo significa che hanno bisogno di uno strumento dinamico
in grado di svolgere numerose attività.

Obiettivo del documento è fornire la descrizione di una piattaforma \textit{cloud based} in grado di risolvere questa necessità.
Essendo le esigenze di ogni azienda uniche nel loro genere si è cercato di realizzare un software in grado
di erogare alcune delle funzionalità tipicamente richieste come la gestione degli utenti oppure l'invio di notifiche.
Questo significa che questa piattaforma software non sarà un sistema autonomo e pronto ad essere utilizzato in un contesto specifico ma
è da considerare come le fondamenta di un sistema più complesso che verrà costruito in base alle esigenze del cliente.

La realizzazione della piattaforma è avvenuta durante il periodo di tirocinio presso Fama Labs,
un'azienda software erogatrice di servizi cloud su commissione.
Questo ha permesso alla azienda ospitante di ottenere un prodotto sul quale costruire nuove soluzioni per i propri clienti
permettendo di ridurne costi e tempistiche.

\section{Panoramica}
La restante parte del documento vuole analizzare la piattaforma realizzata, le tecnologie e il metodo di lavoro.

Il capitolo successivo fornisce una panoramica del sistema software
e verrà presentato il metodo di lavoro applicato. Nel terzo capitolo verranno presentati
gli strumenti tecnologici utilizzati per lo sviluppo della piattaforma. Il quarto capitolo
analizza nel dettaglio i componenti principali della piattaforma mentre il quinto
vuole soffermarsi sulle fasi di integrazione e deployment del software.
