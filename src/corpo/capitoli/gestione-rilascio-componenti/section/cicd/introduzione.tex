\subsection{Introduzione}
CI/CD è un metodo per la distribuzione frequente di software ai clienti, che prevede l'utilizzo di meccanismi
di automazione nelle varie fasi di sviluppo del prodotto che si vuole rilasciare.
L'acronimo CI/CD ha vari significati ma solitamente fa riferimento a dei concetti di integrazione, distribuzione
e deployment continuo.

In particolare "CI" fa riferimento all'integrazione continua. Questo è un processo di automazione
utile agli sviluppatori in cui le nuove modiche apportate al codice sorgente del software vengono
sottoposte a controlli prima di essere pubblicate nel repository condiviso dell'organizzazione.

Il termine "CD" può assumere un duplice significato: distribuzione continua e/o deployment continuo.
Solitamente il processo di distribuzione continua porta alla pubblicazione del nuovo software sul repository condiviso dell'organizzazione
mentre il deployment continuo fa riferimento al suo rilascio automatico all'ambiente di produzione e quindi al cliente.
Questi sono concetti correlati e spesso vengono usati in modo intercambiabile in quanto fanno riferimento
alla automatizzazione delle procedure successive a quelle previste dall'integrazione continua.

Il metodo CI/CD implica quindi la definizione di una serie di processi automatici che devono
essere eseguiti per fornire una nuova versione del software al cliente in maniera rapida e sicura.
Nella piattaforma è stato possibile applicare questa metodologia grazie alla definizione di una pipeline CI/CD che ha
introdotto l'automazione nella metodologia di lavoro adottata nello sviluppo
della piattaforma, permettendo di migliorare l'efficenza e la rapidità con cui le modifiche vengono
integrate e rilasciate al cliente.