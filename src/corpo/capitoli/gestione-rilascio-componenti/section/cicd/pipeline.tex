\subsection{Pipeline CI/CD}
La pipeline CI/CD implementata nella piattaforma suddivide i processi di integrazione e deployment continuo
in un sottoinsieme di attività distinte.

Il flusso di integrazione continua è stato progettato con l'obiettivo di favorire la collaborazione
contemporanea di più sviluppatori sul singolo prodotto mantenendo però un elevato livello di indipendenza
tra loro. Ciò significa che ogni sviluppatore apporta modifiche, in modo indipendente, al software
e le pubblica nel repository condiviso. Tuttavia questa modalità di integrazione
potrebbe portare alla generazione di conflitti fra le modifiche apportare dai vari sviluppatori.

Per risolvere questo problema è stato deciso di implementare un processo di convalida
tramite il quale è possibile garantire la compatibilità fra gli aggiornamenti apportati al software dai membri
del team.
Questo processo viene eseguito ogni volta che uno sviluppatore effettua la pubblicazione dei \textit{commit} sul repository
condiviso e, nel nostro caso, avvia la compilazione del software e l'esecuzione  di unit e integration test.
Questo permette di verificare che le nuove modifiche non introducano problemi nel software.

Il flusso di deployment continuo è invece finalizzato ad automatizzare il rilascio di software pronto ad essere
distribuito nell'ambiente di produzione.
Nello sviluppo della piattaforma il processo di deployment continuo prevede la creazione di una \textit{build production ready}
e la sua pubblicazione sul \textit{production server}. Questo processo automatico ha il vantaggio di rendere
attiva una modifica apportata da uno sviluppatore poco dopo che è stata scritta. Ciò permette di ricevere un
feedback immediato dal cliente e con una cadenza più costante; potendo così apportare
le dovute modifiche al software prima di procedere all'implementazione di nuove funzionalità.

Nella piattaforma queste procedure di automazione sono state realizzate sfruttando Jenkins, un \textit{automation server}
che permette di definire un flusso di operazioni da eseguire per supportare i processi di integrazione e deployment continuo.
La pipeline viene eseguita ogni volta che un commit viene pubblicato nel repository condiviso sulla piattaforma GitLab ed
è stata suddivisa in tre fasi: \textit{build}, \textit{test} e \textit{deploy}.
La fase di \textit{build} prevede il checkout del codice dal repository remoto con la successiva build dei componenti attraverso lo strumento Docker Compose.
La fase di \textit{test} viene utilizzata per validare le modidiche apportate al codice. Al termine si ha la fase di \textit{deploy} in cui vengono pubblicate
le immagini sul registro privato e vengono installati gli strumenti necessari sul \textit{production server} per eseguire la piattaforma. Anche in questo caso si fa
affidamento a Docker Compose per mettere in esecuzione i container sul \textit{production server}.
In caso di successo o fallimento della pipeline è stato inoltre predisposto un meccanismo di messagistica per notificare
gli sviluppatori riguardante l'esito della procedura.

Questo processo ha permesso di rendere lo sviluppo più efficente e ha inoltre contribuito ad aumentare
la sicurezza e la stabilità del software.