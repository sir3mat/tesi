\section{Microservizio Doc}
\subsection{Descrizione generale}
Il microservizio Doc rende accessibile la documentazione della piattaforma.
L'erogazione di questo servizio avviene grazie a un file conforme alla specifica OpenAPI \cite{openAPI},
un insieme di direttive che permettono a una macchina di descrivere, produrre e utilizzare
un servizio REST.

In particolare questo microservizio fornisce una interfaccia interattiva realizzata con
SwaggerUI \cite{SwaggerUI}, una tecnologia che genera un client che permette di comunicare con l'API direttamente
dal browser. L'interfaccia è accessibile attraverso l'API Web che si comporta
come API gateway, andando a fare il proxy delle richieste verso il microservizio Doc.

Durante lo sviluppo della piattaforma la possibilità di interagire con l'API Web in modo rapido ha permesso
di testare direttamente il comportamento degli endpoint. Inoltre il file di specifica, erogato dal microservizio,
potrà poi essere utilizzato dal team frontend per ridurre i tempi necessari per integrare l'API nella applicazione client
attravero Swagger Codegen \cite{SwaggerCodegen}, un generatore di librerie client.