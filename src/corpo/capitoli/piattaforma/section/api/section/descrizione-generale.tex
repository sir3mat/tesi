\subsection{Descrizione generale}
L'API web è il componente della piattaforma che permette di gestire le richieste degli utenti attraverso una interfaccia REST.

Essendo la piattaforma ideata come rappresentazione di una base riutilizzabile in vari contesti aziendali sono stati implementati alcuni dei servizi
tipicamente richiesti come l'autenticazione e l'autorizzazione degli utenti, l'accesso a servizi demo sotto previa registrazione, l'invio di notifiche e
la gestione delle informazioni dell'utente. Questo permetterà future implementazioni personalizzate riducendo notevolmente i costi e i tempi di sviluppo potendo così dedicare
tutte le risorse sulle funzionalità richieste dalla clientela.

L'applicazione è basata su Node ed è stato utilizzato il framework Nest.
La comunicazione tra client e API web verrà gestita con la suite di protocolli TCP/IP e in particolare verrà usato il protocollo HTTPS per gestire
le richieste e le risposte.
Si interfaccia con il microservizio del Mailer, attraverso un \textit{message broker} RabbitMQ, per gestire l'invio di email e con il database server, basato su MongoDB, per gestire i dati necessari
per erogare i servizi richiesti. Inoltre interagisce con il microservizio Doc per mostrare la documentazione degli endpoint della piattaforma attraverso
una interfaccia interattiva.
