\subsection{Mail module}
Il modulo mail ha la responsabilità di gestire l'invio di messaggi al \textit{message broker} per richiedere
l'invio di una mail.

Viene esposto un servizio che permette ad altri componenti di richiedere l'invio di mail al microservizio mailer.
In particolare è presente una interfaccia che definisce i campi che devono essere presenti
nel messaggio che verrà recapitato al microservizio mailer per permettergli di capire quale
template renderizzare e come inviare la mail.

Questo servizio si occuperà poi di incapsulare tutti i dettagli necessari per poter comunicare con il
\textit{message broker}. In particolare viene utilizzato un modulo wrapper della libreria \textit{amqplib} che permette di comunicare
con il \textit{message broker} utilizzato nella piattaforma.
Con l'utilizzo di questo modulo è possibile definire un meccanismo interno per inviare le mail
attraverso l'uso di una interfaccia. Inoltre si mantiene il componente aperto alle estensioni
in caso si voglia utilizzare un protocollo diverso per comunicare con il \textit{message broker}.