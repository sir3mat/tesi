\section{Gestione codice condiviso}
In questa sezione verranno presentate le tecnologie utilizzate per supportare le operazioni
del team di sviluppo e aumentarne la produttività.
La motivazione principale che ha portato alla scelta delle tecnologie di seguito elencate è il \textit{know-how} aziendale.

\subsection{Git}
Git \cite{Git} è un sistema di controllo versione distribuito (DVCS), gratuito e open source.
Nato nel 2005 grazie all'operato della comunità Linux e di Linus Torvalds, il creatore di Linux, è a oggi uno strumento che
permette a un team di sviluppo di collaborare in modo facile, veloce ed efficiente su grandi progetti.

Essendo un sistema di controllo versione permette di registrare, nel tempo, i cambiamenti a un file o a una serie di file,
così da poter richiamare una specifica versione in un secondo momento. \cite{GitPro}.
Ciò offre numerosi vantaggi tra cui quello di poter ripristinare un file o un intero progetto a uno stato precedente a una modifica,
vedere chi e quando ha cambiato qualcosa e recuperare file cancellati.
Inoltre offre un sistema di ramificazione (branching) per lo sviluppo non lineare.

È stato utilizzato nel contesto dello sviluppo della applicazione attraverso GitLab, una piattaforma web open source che consente
la gestione di repository Git.

\subsection{Monorepo}
Una \textit{monorepo}\cite{Google-Monorepo} (\textit{mono repository}) è una strategia di sviluppo software dove il codice
relativo a vari progetti o applicazioni risiede in uno stesso luogo, solitamente una \textit{repository} condivisa
gestita con un sistema di controllo versione.
Si contrappone al modello di singola \textit{repository} per progetto.

Di seguito alcuni vantaggi offerti dall'utilizzo della strategia monorepo.\cite{Monorepo}
\begin{itemize}
      \itemsep0em
      \item Gestione semplificata delle dipendenze: le singole applicazioni condividono le stesse dipendenze.
            Questo stimola la coesione negli strumenti e librerie utilizzati e
            aumenta la produttività degli sviluppatori.
      \item Organizzazione semplificata: è possibile progettare una
            gerarchia logica fra i progetti per renderli logicamente
            relazionati anche all'interno della repository.
      \item Semplificazione processi di release: esiste una pipeline condivisa per la build e il deploy del sistema.
      \item Condivisione codice: è possibile utilizzare delle librerie interne per condividere codice fra le varie applicazioni riducendo
            così la duplicazione e favorendo la creazione di codice migliore.
      \item Standard e convenzioni: è possibile introdurre un utilizzo coeso relativo alle convenzioni di sviluppo e all'utilizzo delle tecnologie
            usate per testing, debug e build dei progetti. Queste permette di creare codice consistente e di qualità.

\end{itemize}
L'utilizzo di una monorepo permette quindi di creare piattaforme complesse, composte da applicazioni e microservizi, in modo
coeso, dal punto di vista degli strumenti e tecnologie usate, e permettendo agli sviluppatori di avere le conoscenze adatte per lavorare
in modo efficiente ed efficace ai vari progetti della organizzazione.
Nonostante ciò l'utilizzo di questa strategia introduce un certo livello di rigidità nella gestione delle applicazioni e un
grado di complessità, proporzionale alle dimensioni della monorepo, nell'introduzione di nuovi membri nel team di sviluppo.

Nello sviluppo della piattaforma è stato deciso di utilizzare questa strategia e ciò ha permesso
al team di sviluppo di operare in modo efficiente e coeso.

\subsubsection{Nx}
Nx \cite{Nx} è uno strumento open source, creato dal team Nrwl \cite{Nrwl}, che permette a una organizzazione
di gestire una monorepo.
Il suo funzionamento è basato sulla CLI di Angular, un framework per lo sviluppo di applicazioni front-end,
ed è composto da un insieme di strumenti che permettono di gestire in modo flessibile una monorepo per progetti basati
su Node.js, React o Angular. Alcune delle funzionalità offerte sono la generazione automatica della gerarchia di file e cartelle
per generare una nuova applicazione o libreria, possibilità di eseguire script in un contesto globale o relegato alle singole applicazioni
e meccanismi di ottimizzazione per le fasi di build e test grazie all'utilizzo dei un meccanismo di caching.

Nello sviluppo della piattaforma è stato decisa l'introduzione questo strumento in quanto permette di semplificare
notevolmente la fase di design della monorepo stessa.