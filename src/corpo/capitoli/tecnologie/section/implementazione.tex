\section{Implementazione}

\par
In questa sezione verranno descritti gli strumenti utilizzati per implementare i componenti che
permettono alla piattaforma di erogare i propri servizi.

\subsection{Ambiente di sviluppo}

\par
La scelta dell'ambiente di sviluppo, ovvero l'insieme di strumenti e tecnologie che permettono di sviluppare codice sorgente, è stata imposta dal know-how aziendale.
\subsubsection{Node.js}

\par
Node.js è un ambiente runtime JavaScript open-source e multipiattaforma.
\par
Le caratteristiche fondamentali sono: l'esecuzione dell'engine V8, sviluppato da Google, che permette di compilare ed eseguire codice JavaScript al di fuori di browser web,
l'uso di un insieme di primitive I/O asincrone di tipo non bloccante e l'esecuzione di applicazioni su un solo processo, senza generazione di nuovi thread per ogni richiesta.
Pertanto quando si deve eseguire una operazione I/O, come una richiesta ad un web server, Node.js non blocca il thread, mettendo in attesa la CPU, ma, al contrario,
la lascia libera di portare avanti altri compiti e si occuperà di ripristinare l'operazione non appena arriverà una risposta.
\par
Grazie a queste peculiarità è possibile realizzare applicazioni performanti in grado di gestire connessioni concorrenti con un singolo server, senza introdurre la complessità
logica legata alla gestione della concorrenza fra thread.
\par
In questo ambiente è poi possibile utilizzare lo standard ECMAScript nelle sue varie versioni in modo flessibile in quanto è possibile modificare il set di funzionalità abilitate,
potendo così adattarsi al meglio nei vari contesti di utilizzo.
\par
Infine, Node.js permette anche di aumentare la produttività di un team di sviluppo perchè fornisce agli sviluppatori front-end,
che conoscono il linguaggio JavaScript, la possibilità di sviluppare codice \textit{server-side}; senza dover imparare un linguaggio del tutto nuovo.
\par
Grazie alle sue caratteristiche Node.js risulta essere un'ottimo strumento per lo sviluppo di servizi web.

\subsection{Linguaggio}
\subsubsection{Typescript}

\subsection{Framework}
\subsubsection{Nest.js}

\subsection{Testing}
\subsubsection{Jest}

