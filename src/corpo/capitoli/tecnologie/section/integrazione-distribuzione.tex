\section{Integrazione e rilascio del software}
In questa sezione sono descritte le tecnologie utilizzate nelle fasi di integrazione e rilascio del software.

\subsection{Docker}
Docker \cite{Docker} è una tecnologia open source per creare, distribuire e gestire applicazioni sottoforma di \textit{container}.
Un container Docker è un'ambiente isolato in cui può essere messa in esecuzione una applicazione.
Più nello specifico si crea un container basato su Linux in cui viene caricata un'immagine di una applicazione con le relative dipendenze.

Vantaggio offerto dalla piattaforma Docker è il fatto che questi container sono isolati li uni dagli altri
e ciò permette di poterli eseguire in sicurezza e indipendentemente.
Altra nota positiva è la leggerezza. Infatti i container Docker, a differenza delle macchine virtuali, non necessitano
di un hypervisor ma vengono eseguiti direttamente dal kernel della macchina host.

L'utilizzo dei container permette poi di disaccopiare le fasi di rilascio e deployment delle applicazioni ed offre la garanzia
di avere un'ambiente in cui si ha la certezza che il comportamento della applicazione sarà quello previsto. Questo risulta
vantaggioso anche per la produttività degli sviluppatori che possono così condividere
il proprio lavoro più facilmente.

Questa tecnologia è stata usata nello sviluppo della piattaforma per supportare le operazioni di integrazione e distribuzione.
Viene inoltre utilizzato per effettuare il deployment, ovvero il rilascio dell'applicazione al reparto di produzione.

\subsection{Jenkins}
Jenkis \cite{Jenkins} è uno strumento open source che fa parte della categoria degli \textit{automation server}, ovvero applicazioni server utilizzate
per automatizzare le task relative alle fasi di build, test, rilascio e deploy del software.

Basa il suo funzionamento su una pipeline, ovvero un insieme di step, detti \textit{stage}, da eseguire in sequenza per portare a termine un processo.
Ogni step potrà contenere a sua volte delle istruzioni, detti \textit{jobs}, da eseguire per portare a termine un dato compito quale potrebbe essere la build, i test o il deployment.
L'esecuzione della pipeline può avvenire in varie modalità pianificata o utilizzando dei trigger, come ad esempio il commit su una repository o il fallimento di un test.
