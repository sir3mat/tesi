\section{Gestione dati}
In questa sezione verranno descritti gli strumenti utilizzati per la gestione e l'aggiornamento delle informazioni
necessare per il funzionamento della piattaforma.
La motivazione principale che ha portato alla scelta delle tecnologie di seguito elencate è il \textit{know-how} aziendale.
\subsection{Database}
\subsubsection{MongoDB}
MongoDB\cite{MongoDB} è un database management system (DBMS) che offre la possibilità di gestire database NoSQL
\footnote{
    dall'inglese \textit{Not only SQL}; non solo SQL.
} basati sull'utilizzo di documenti flessibili, al posto delle tabelle, per processare e salvare dati in vario formato.\cite{IBM-MongoDB}
L'unità fondamentale in un database basato su MongoDB sono i documenti. Questi rappresentano l'informazioni da memorizzare.
Sono formattati in formato \textit{Binary JSON} (BSON) e possono includere varie tipologie di dati che vanno dai comuni
valori numerici o stringhe sino ai più complessi oggetti innidati e liste.
La particolarità di questi documenti è che non hanno uno schema rigido: possono subire delle modifiche nel tempo permettendo
ai dati di adattarsi alla applicazione e non viceversa. Questo offre un notevole vantaggio per lo sviluppatore
in quanto può strutturare i dati nella maniera più consona per facilitarne l'utilizzo.
Questi documenti vengono poi raccolti in collezioni che vengono usate come archivi per documenti facenti parte di una stessa categoria di informazione.
Tutti questi dati possono poi essere sottoposti a query e ad aggregazioni più o meno complesse fornendo una API in grado di fornire operazioni in modo molto efficente.

MongoDB risulta essere anche molto performante grazie alla possibilità di immagazzinare documenti innestati, riducendo l'attività I/O del sistema DB,
e all'utilizzo del meccanismo di indicizzazione che permette di effettuare query molto velocemente.

Altre caratteristiche fondamentali di questa tecnologia sono l'alta disponibilità e la possiblità di scalare orizzontalmente il sistema.
La prima è ottenuta con l'uso dei \textit{replica set}, che permettono di gestire database replicati, in modo da garantire ridondanza.
La seconda è invece realizzabile con il metodo dello \textit{Sharding} distribuito che permette
la distribuzione del carico computazionale, per la gestione delle richieste, su più server; fornendo così una esecuzione più efficente delle operazioni rispetto all'utilizzo di una solo server.


