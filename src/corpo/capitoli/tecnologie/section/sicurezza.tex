\section{Sicurezza}
In questa sezione verranno descritti i meccanismi di autenticazione e autorizzazione scelti per
garantire un adeguato livello di sicurezza agli utenti.

\subsection{Autenticazione}
L'autenticazione è un metodo che permette di identificare gli utenti sulla base di un set di credenziali e
verificare che questi forniscano le giuste informazioni per accedere alla applicazione.

L'implementazione presente all'interno della piattaforma si basa su una coppia di credenziali: email e password.
Per garantire una archiviazione sicura delle password è stato deciso di applicare una policy basata sull'uso di \textit{hash} e \textit{salt}.


- spiegazione di hash -
- spiegazione di salt -
- che problema risolve la loro introduzione -


\subsection{Autorizzazione}
L'autorizzazione è un metodo che permette di garantire la riservatezza e la disponibilità dei dati.
Attraverso l'implementazione di policy è infatti possibile definire dei protocolli che permettono di restringere l'accesso alle risorse,
rendendole disponibili solo agli utenti autorizzati.

Le policy definite all'interno della piattaforma si basano sul protocollo di autorizzazione Oauth 2.0 \cite{rfc6749} e sull'utilizzo di un
modello di controllo sugli accessi basato sui ruoli dei singoli utenti.


\subsubsection{OAuth2.0}
- definizione oauth2.0 -
- descrizione categoria utilizzata da noi con riferimento alle altre come nota -
- che problema vuole risolvere -

\subsubsection{JWT}
- definizione jwt -
- struttura -
- cifratura -