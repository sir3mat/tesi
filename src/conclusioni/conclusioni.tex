\chapter{Conclusioni}
\section{Valutazioni complessive}
La piattaforma software è stata realizzata con lo scopo di divenire una piattaforma basata sul cloud
in grado di soddisfare le esigenze di una vasta gamma di clienti attraverso l'implementazione
di funzionalità personalizzate. Durante l'analisi del prodotto realizzato sono state descritte le tecnologie
e gli strumenti utilizzati per implementare i componenti fondamentali della piattaforma nonchè
il metodo di lavoro e gli automatismi adottati per erogare agilmente il software al cliente.

Un esempio di utilizzo della piattaforma risiede nella Medical Adaptive Platform (MAP), un nuovo servizio
cloud based erogato dall'azienda Fama Labs. Questo è una suite di soluzioni digitali per le Società Scientifiche,
aziende biofarmaceutiche e professionisti sanitari che offre servizi per gestire l'amministrazione della società,
organizzare eventi di formazione a distanza, corsi e congressi.

La piattaforma realizzata verrà inoltre sfruttata dall'azienda Fama Labs in una applicazione interna
per il management dei servizi erogati e delle risorse.


\section{Sviluppi futuri}
Sviluppi futuri riguardano il tema della scalabilità e un aumento del livello di disaccoppiamento tra i servizi erogati
per favorire maggiormente una architettura orientata ai microservizi.
In particolare, per quanto riguarda la scalabilità, verranno introdotte soluzioni di orschestrazione
dei container basate su tecnologie come Docker Swarm o Kubernetes con lo scopo di poter fornire maggiori opzioni al
cliente in fase contrattuale. Infatti in questa prima versione della piattaforma si ha che tutti i componenti risiedono in un unico host
e nonostante sia possibile scalare i singoli servizi i limiti dati da questa situazione sono evidenti.
Verranno quindi progettate soluzioni alternative in cui in componenti verranno messi in esecuzione su host separati per
poter offrire un livello di scalabilità orizzontale più elevato sui servizi di maggiore interesse per il business del cliente.
